\documentclass{article}
\usepackage[utf8]{inputenc}
\usepackage[polish]{babel}
\usepackage[T1]{fontenc}
\usepackage{graphicx}
\usepackage{booktabs}
\usepackage{csvsimple}
\usepackage{longtable}
\usepackage[margin=0.5in]{geometry}
\usepackage{textcomp}
\usepackage{pgfplots}
\pgfplotsset{width=10cm,compat=1.9}


\title{Spadek swobodny}
\author{Kamila Grońska}
\date{December 2022}

\begin{document}
\maketitle
\section{Wprowadzenie teoretyczne}
\begin{center}
\begin{equation}
 h(t) = h_o-{gt^2\over 2}   
\end{equation}
\end{center}

\section{Przebieg eksperymentu }
\textit{Jak widać na rysunku\ref{fig:Schemat doświadczenia}}


\begin{figure}[h]
\begin{center}
    
\includegraphics[width=0.4\textwidth]{swobodny-spadek-przyspieszenie-ziemskie2.png}
\caption{Schemat doświadczenia}
\label{fig:Schemat doświadczenia}
\end{center}
\end{figure}

\newpage
\section{Wyniki pomiarów}


\csvautolongtable{tabela.csv} 


\section{Wnioski}

\end{document}
