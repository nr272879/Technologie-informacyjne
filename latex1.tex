\documentclass{article}
\usepackage[utf8]{inputenc}
\usepackage[polish]{babel}
\usepackage[T1]{fontenc}
\usepackage{graphicx}
\usepackage{booktabs}


\title{Spadek swobodny}
\author{Kamila Grońska}
\date{December 2022}

\begin{document}
\maketitle
\section{Wprowadzenie teoretyczne}
\begin{center}
\begin{equation}
 h(t) = h_o-{gt^2\over 2}   
\end{equation}
\end{center}

\section{Przebieg eksperymentu }
\textit{Jak widać na rysunku\ref{fig:Schemat doświadczenia}}


\begin{figure}[h]
\begin{center}
    
\includegraphics[width=0.4\textwidth]{swobodny-spadek-przyspieszenie-ziemskie2.png}
\caption{Schemat doświadczenia}
\label{fig:Schemat doświadczenia}
\end{center}
\end{figure}

\section{Wyniki pomiarów}
\begin{tabular}{l|c|r}
\toprule
t[s]  &	s[m]	&	s2[m]\\
\midrule
0,00  &	0,00	&	0,00\\
0,10 &	0,05	&	0,26\\
0,20 &	0,20	&	-3,44\\
0,30 &	0,44	&	2,70\\
0,40 &	0,78	&	-3,28\\
0,50 &	1,23	&	2,26\\
0,60 &	1,76	&	2,37\\
0,70 &	2,40	&	4,74\\
0,80 &	3,14	&	1,00\\
0,90 &	3,97	&	3,75\\
1,00 &	4,90	&	7,27\\
1,10 &	5,93	&	1,61\\
1,20 &	7,06	&	7,85\\
1,30 &	8,28	&	13,92\\
1,40 &	9,60	&	11,82\\
1,50 &	11,03	&	14,65\\
1,60 &	12,54	&	14,97\\
1,70 &	14,16	&	17,87\\
1,80 & 	15,88	&	9,51\\
1,90 &  17,69	&	11,80\\
2,00 &	19,60	&	16,44\\
\bottomrule
\end{tabular}

\section{Wnioski}

\end{document}
